% Experimental Design and Configuration
% LaTeX structure for thesis chapter on experimental methodology

\section{Experimental Design and Methodology}

This section describes the comprehensive experimental framework designed to evaluate the proposed transductive constraint-based optimization approach for student dropout prediction. The experiments were systematically configured to explore a wide range of neural network architectures, constraint settings, and hyperparameter combinations, enabling thorough empirical validation and comparative analysis.

\subsection{Automated Configuration Management}

To facilitate large-scale experimentation and ensure reproducibility, we developed an automated configuration management system that generates, tracks, and executes experiments systematically. This approach offers several key advantages:

\begin{itemize}
    \item \textbf{Reproducibility}: Each experiment configuration is stored as a JSON file containing all hyperparameters, model specifications, and constraint settings, ensuring exact replication of results.
    \item \textbf{Scalability}: The system supports parallel execution of hundreds of experiments with automatic status tracking (pending, running, completed).
    \item \textbf{Model Reuse}: Warmup phase models are cached using hash-based identifiers, allowing multiple constraint configurations to share the same pre-trained base model, significantly reducing computational overhead.
    \item \textbf{Systematic Coverage}: By generating all combinations of models, constraints, and hyperparameters programmatically, we ensure comprehensive exploration of the experimental space without manual oversight errors.
\end{itemize}

The configuration system generated a total of \textbf{640 unique experiment configurations}, combining 5 neural network architectures, 8 constraint scenarios, and 16 hyperparameter variations per model-constraint pair.

\subsection{Neural Network Architectures}

To evaluate the generalizability of our constraint-based optimization approach across different model complexities and architectural paradigms, we tested five diverse neural network architectures, ranging from simple feedforward networks to sophisticated deep learning models inspired by computer vision research.

\subsubsection{BasicNN: Baseline Feedforward Network}

A simple feedforward neural network serving as our baseline architecture. This model provides a reference point for understanding the contribution of architectural complexity to constraint satisfaction and predictive accuracy.

\begin{itemize}
    \item \textbf{Architecture}: Sequential fully-connected layers with batch normalization
    \item \textbf{Hidden Layers}: Two layers with dimensions [128, 64]
    \item \textbf{Activation}: ReLU activation functions
    \item \textbf{Regularization}: Dropout layers (configurable rate) after each hidden layer
    \item \textbf{Output}: 3-class softmax for (Dropout, Enrolled, Graduate)
    \item \textbf{Motivation}: Establishes baseline performance for comparison with more complex architectures
\end{itemize}

\subsubsection{ResNet56: Residual Network}

Inspired by the ResNet architecture \cite{he2016deep}, this model incorporates residual skip connections to facilitate gradient flow in deeper networks and enable identity mappings.

\begin{itemize}
    \item \textbf{Architecture}: Residual blocks with skip connections
    \item \textbf{Hidden Dimensions}: Four layers [256, 256, 128, 128]
    \item \textbf{Skip Connections}: Identity shortcuts bypassing transformation layers
    \item \textbf{Regularization}: Batch normalization and dropout
    \item \textbf{Motivation}: Tests whether residual connections improve constraint-aware learning by providing alternative gradient pathways
\end{itemize}

\subsubsection{DenseNet121: Densely Connected Network}

Based on the DenseNet architecture \cite{huang2017densely}, this model implements dense connectivity patterns where each layer receives inputs from all preceding layers through concatenation.

\begin{itemize}
    \item \textbf{Architecture}: Three dense blocks with 4 layers each
    \item \textbf{Growth Rate}: 32 features added per layer
    \item \textbf{Connectivity}: Feature concatenation from all previous layers
    \item \textbf{Transition Layers}: Compression between dense blocks
    \item \textbf{Motivation}: Evaluates whether feature reuse and dense connectivity enhance the model's ability to learn constraint-satisfying representations
\end{itemize}

\subsubsection{InceptionV3: Multi-Scale Feature Extraction}

Adapted from the Inception architecture \cite{szegedy2016rethinking}, this model processes inputs through parallel pathways with different depths to capture multi-scale feature representations.

\begin{itemize}
    \item \textbf{Architecture}: Four parallel branches with different depths
    \item \textbf{Branch Depths}: 1, 2, 3, and 4 layers respectively
    \item \textbf{Feature Aggregation}: Concatenation of all branch outputs
    \item \textbf{Motivation}: Tests whether multi-scale feature extraction improves prediction quality under constraints
\end{itemize}

\subsubsection{VGG19: Deep Sequential Network}

Inspired by the VGG architecture \cite{simonyan2014very}, this model uses a simple but deep sequential structure to learn hierarchical feature representations.

\begin{itemize}
    \item \textbf{Architecture}: Deep sequential stack of fully-connected layers
    \item \textbf{Depth}: 6 hidden layers
    \item \textbf{Layer Pattern}: Uniform layer structure with consistent width
    \item \textbf{Motivation}: Evaluates whether increased depth alone (without architectural innovations) benefits constraint-based optimization
\end{itemize}

\subsection{Constraint Configuration Scenarios}

Constraints in our problem represent enrollment capacity limitations at both global (entire institution) and local (per-course) levels. We systematically varied constraint tightness to evaluate model behavior under different restriction levels. Each constraint is expressed as a percentage multiplier of the test set class distribution, scaled by a factor of 10 to represent realistic capacity limitations.

The constraint pairs are expressed as (local percentage, global percentage), where:
\begin{itemize}
    \item \textbf{Local percentage}: Proportion of per-course capacity for dropout and enrollment predictions
    \item \textbf{Global percentage}: Proportion of institution-wide capacity for dropout and enrollment predictions
    \item \textbf{Graduate class}: Always unlimited (serves as unconstrained category)
\end{itemize}

We tested eight constraint scenarios, ordered from most permissive to most restrictive:

\begin{table}[h]
\centering
\caption{Constraint Configuration Scenarios}
\label{tab:constraints}
\begin{tabular}{cccc}
\hline
\textbf{Scenario} & \textbf{Local \%} & \textbf{Global \%} & \textbf{Restriction Level} \\
\hline
C1 & 0.9 & 0.8 & Very Permissive \\
C2 & 0.9 & 0.5 & Permissive (Global Tight) \\
C3 & 0.8 & 0.7 & Moderate-Permissive \\
C4 & 0.8 & 0.2 & Moderate (Global Very Tight) \\
C5 & 0.7 & 0.5 & Moderate \\
C6 & 0.6 & 0.5 & Moderate-Restrictive \\
C7 & 0.5 & 0.3 & Restrictive \\
C8 & 0.4 & 0.2 & Very Restrictive \\
\hline
\end{tabular}
\end{table}

\textbf{Constraint Interpretation Example:}
For scenario C8 (0.4, 0.2) with a test set containing 400 students where 100 are true dropouts:
\begin{itemize}
    \item Global dropout constraint: $\lfloor 100 \times 0.2 / 10 \rfloor = 2$ students
    \item Local dropout constraint per course: $\lfloor n_{\text{course}} \times 0.4 / 10 \rfloor$ students
\end{itemize}

This range of constraints allows us to evaluate:
\begin{enumerate}
    \item How models perform under mild vs. severe capacity restrictions
    \item Whether the proposed optimization approach can satisfy constraints across all tightness levels
    \item The trade-off between prediction accuracy and constraint satisfaction as restrictions increase
    \item Comparative performance of different architectures under varying constraint pressure
\end{enumerate}

\subsection{Hyperparameter Exploration}

To ensure robust evaluation and identify optimal training configurations, we conducted comprehensive hyperparameter sensitivity analysis across four key dimensions. Each model-constraint pair was trained under multiple hyperparameter regimes, totaling 16 configurations per pair.

\subsubsection{Base Hyperparameters}

The default configuration used across all experiments unless explicitly varied:

\begin{table}[h]
\centering
\caption{Base Hyperparameter Configuration}
\label{tab:base_hyperparams}
\begin{tabular}{lll}
\hline
\textbf{Parameter} & \textbf{Value} & \textbf{Description} \\
\hline
Learning Rate ($\eta$) & 0.001 & Adam optimizer step size \\
Dropout Rate & 0.3 & Regularization dropout probability \\
Batch Size & 64 & Mini-batch size for SGD \\
Hidden Dimensions & [128, 64] & Layer widths for BasicNN \\
Max Epochs & 10,000 & Maximum training iterations \\
Warmup Epochs & 250 & Pre-constraint training phase \\
$\lambda_{\text{global}}$ (initial) & 0.01 & Global constraint weight \\
$\lambda_{\text{local}}$ (initial) & 0.01 & Local constraint weight \\
$\lambda$ step size & 0.01 & Adaptive lambda increment \\
Constraint threshold & $10^{-6}$ & Convergence criterion \\
\hline
\end{tabular}
\end{table}

\subsubsection{Learning Rate Sensitivity Analysis}

Learning rate critically affects both convergence speed and final solution quality, particularly when balancing multiple objectives (accuracy + constraints). We tested five learning rates spanning three orders of magnitude:

\begin{itemize}
    \item \textbf{Values tested}: $\{10^{-4}, 5 \times 10^{-4}, 10^{-3}, 5 \times 10^{-3}, 10^{-2}\}$
    \item \textbf{Motivation}: Identify optimal step size for joint optimization of cross-entropy loss and constraint losses
    \item \textbf{Hypothesis}: Moderate learning rates ($10^{-3}$ to $10^{-2}$) may better balance gradient contributions from competing objectives
    \item \textbf{Configurations per model-constraint pair}: 5
\end{itemize}

\subsubsection{Dropout Regularization Analysis}

Dropout prevents overfitting but may also affect the model's ability to learn constraint-satisfying patterns. We systematically varied dropout intensity:

\begin{itemize}
    \item \textbf{Values tested}: $\{0.1, 0.2, 0.3, 0.4, 0.5\}$
    \item \textbf{Motivation}: Determine optimal regularization strength for transductive learning with constraints
    \item \textbf{Hypothesis}: Higher dropout may improve generalization but could hinder constraint satisfaction by limiting model capacity
    \item \textbf{Configurations per model-constraint pair}: 5
\end{itemize}

\subsubsection{Batch Size Analysis}

Batch size affects gradient estimation quality and training dynamics, with potential implications for constraint-aware optimization:

\begin{itemize}
    \item \textbf{Values tested}: $\{32, 64, 128, 256, 512\}$
    \item \textbf{Motivation}: Explore trade-offs between gradient noise (small batches) and computational efficiency (large batches)
    \item \textbf{Hypothesis}: Larger batches provide more stable constraint loss estimates but may reduce exploration capacity
    \item \textbf{Configurations per model-constraint pair}: 5
\end{itemize}

\subsubsection{Standard Configuration}

In addition to the three sensitivity analyses above, each model-constraint pair was trained with the base hyperparameters as a reference point:

\begin{itemize}
    \item \textbf{Configurations per model-constraint pair}: 1
    \item \textbf{Purpose}: Establishes baseline performance for hyperparameter comparisons
\end{itemize}

\subsection{Training Methodology}

All experiments followed a two-phase training protocol designed to leverage standard supervised learning before introducing constraint pressures:

\subsubsection{Phase 1: Warmup (Epochs 1--250)}

\begin{itemize}
    \item \textbf{Objective}: Standard cross-entropy minimization on training set
    \item \textbf{Loss function}: $\mathcal{L}_{\text{CE}}(\theta) = -\sum_{i=1}^{N_{\text{train}}} \sum_{c=1}^{3} y_{ic} \log p_{\theta}(y_i = c | x_i)$
    \item \textbf{Purpose}: Build strong feature representations before constraint optimization
    \item \textbf{Caching}: Models are saved with hash-based identifiers for reuse across constraint variations
\end{itemize}

\subsubsection{Phase 2: Constraint Optimization (Epochs 251+)}

\begin{itemize}
    \item \textbf{Objective}: Joint optimization of accuracy and constraint satisfaction
    \item \textbf{Loss function}:
    \begin{align}
    \mathcal{L}_{\text{total}}(\theta) = \mathcal{L}_{\text{CE}}(\theta) + \lambda_{\text{global}} \mathcal{L}_{\text{global}}(\theta) + \lambda_{\text{local}} \mathcal{L}_{\text{local}}(\theta)
    \end{align}
    \item \textbf{Constraint losses}: Rational saturation loss $\mathcal{L}_{\text{constraint}} = \frac{E}{E + K}$ where $E$ = excess predictions, $K$ = constraint limit
    \item \textbf{Adaptive weighting}: $\lambda$ values increase by step size (0.01) whenever constraint loss exceeds threshold ($10^{-6}$)
    \item \textbf{Early stopping}: Training terminates when both global and local constraint losses fall below threshold or max epochs reached
\end{itemize}

\subsection{Experimental Coverage and Completeness}

The experimental design ensures comprehensive coverage of the hypothesis space through systematic combinatorial generation:

\begin{table}[h]
\centering
\caption{Experimental Space Coverage}
\label{tab:coverage}
\begin{tabular}{lcc}
\hline
\textbf{Dimension} & \textbf{Variations} & \textbf{Total Experiments} \\
\hline
Neural Architectures & 5 & -- \\
Constraint Scenarios & 8 & -- \\
Hyperparameter Regimes & 4 & -- \\
\quad $\hookrightarrow$ Standard & 1 & 40 \\
\quad $\hookrightarrow$ Learning Rate & 5 & 200 \\
\quad $\hookrightarrow$ Dropout & 5 & 200 \\
\quad $\hookrightarrow$ Batch Size & 5 & 200 \\
\hline
\textbf{Total Unique Configurations} & & \textbf{640} \\
\hline
\end{tabular}
\end{table}

This exhaustive experimental design enables:

\begin{enumerate}
    \item \textbf{Architectural Comparison}: Direct performance comparison across 5 architectures under identical conditions (40 experiments per architecture)
    \item \textbf{Constraint Robustness}: Evaluation of each architecture across 8 constraint tightness levels (80 experiments per architecture)
    \item \textbf{Hyperparameter Sensitivity}: Systematic analysis of training dynamics across 16 configurations per model-constraint pair
    \item \textbf{Statistical Significance}: Sufficient sample size for meaningful statistical analysis of performance differences
    \item \textbf{Ablation Studies}: Isolation of individual hyperparameter effects through controlled variation
\end{enumerate}

\subsection{Baseline Comparison}

To validate the effectiveness of our transductive constraint optimization approach, we implemented a post-hoc heuristic baseline for comparison:

\subsubsection{Greedy Constraint Heuristic}

\begin{itemize}
    \item \textbf{Method}: Uses cached warmup models (identical to optimization approach) but applies constraints post-hoc through greedy allocation
    \item \textbf{Algorithm}:
    \begin{enumerate}
        \item Obtain prediction probabilities from warmup model (250 epochs, no constraints)
        \item For each class in priority order (Graduate $\rightarrow$ Dropout $\rightarrow$ Enrolled):
        \begin{itemize}
            \item Sort samples by class probability (descending)
            \item Assign top samples respecting both global and local constraints
        \end{itemize}
        \item Assign remaining samples to highest probability class
    \end{enumerate}
    \item \textbf{Purpose}: Establishes performance ceiling for non-constraint-aware training
    \item \textbf{Fair Comparison}: Uses identical warmup models, ensuring architectural and hyperparameter parity
\end{itemize}

\subsection{Evaluation Metrics}

\textbf{TODO: Add your comparative results and analysis here}

% Placeholder for results tables and figures
% Example:
% \begin{figure}[h]
% \centering
% % \includegraphics[width=0.8\textwidth]{figures/accuracy_vs_constraints.pdf}
% \caption{Model accuracy across constraint tightness levels}
% \label{fig:accuracy_constraints}
% \end{figure}

% \begin{figure}[h]
% \centering
% % \includegraphics[width=0.8\textwidth]{figures/architecture_comparison.pdf}
% \caption{Performance comparison across neural architectures}
% \label{fig:architecture_comparison}
% \end{figure}

% \begin{figure}[h]
% \centering
% % \includegraphics[width=0.8\textwidth]{figures/hyperparameter_sensitivity.pdf}
% \caption{Hyperparameter sensitivity analysis}
% \label{fig:hyperparam_sensitivity}
% \end{figure}

\subsection{Reproducibility and Availability}

All experimental configurations, training scripts, and evaluation code are available in the project repository. The automated configuration system ensures:

\begin{itemize}
    \item Complete specification of experimental conditions in JSON format
    \item Deterministic random seed initialization (seed = 42)
    \item Comprehensive logging of training dynamics (per-epoch metrics, constraint satisfaction status)
    \item Version-controlled codebase with clear documentation
\end{itemize}

% Bibliography references
% Add these to your main .bib file:
% @inproceedings{he2016deep,
%   title={Deep residual learning for image recognition},
%   author={He, Kaiming and Zhang, Xiangyu and Ren, Shaoqing and Sun, Jian},
%   booktitle={Proceedings of the IEEE conference on computer vision and pattern recognition},
%   pages={770--778},
%   year={2016}
% }
%
% @inproceedings{huang2017densely,
%   title={Densely connected convolutional networks},
%   author={Huang, Gao and Liu, Zhuang and Van Der Maaten, Laurens and Weinberger, Kilian Q},
%   booktitle={Proceedings of the IEEE conference on computer vision and pattern recognition},
%   pages={4700--4708},
%   year={2017}
% }
%
% @inproceedings{szegedy2016rethinking,
%   title={Rethinking the inception architecture for computer vision},
%   author={Szegedy, Christian and Vanhoucke, Vincent and Ioffe, Sergey and Shlens, Jon and Wojna, Zbigniew},
%   booktitle={Proceedings of the IEEE conference on computer vision and pattern recognition},
%   pages={2818--2826},
%   year={2016}
% }
%
% @article{simonyan2014very,
%   title={Very deep convolutional networks for large-scale image recognition},
%   author={Simonyan, Karen and Zisserman, Andrew},
%   journal={arXiv preprint arXiv:1409.1556},
%   year={2014}
% }
